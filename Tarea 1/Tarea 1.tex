\documentclass[11pt]{article}
    \usepackage{notetaking}

\title{Tarea 1}
\author{Nicholas Mc-Donnell}
\date{03/2018}

\begin{document}
    \maketitle
    \begin{enumerate}
        \item Demuestre que los siguientes polinomios son irreducibles en $\set{Q}[x]$
            \begin{itemize}
                \item $x^3-x-4$
        
                \item $x^3-\frac{3}{2}x+1$
        
                \item $x^4-x^2+1$
            \end{itemize}
            \begin{proof}
                \
                \begin{itemize}
                    \item Por teorema de raíces racionales, y ya que es un polinomio de tercer grado, el polinomio es reducible si y solo si tiene una raíz.
                    \[p(x)=0\implies x\in\{\pm1,\pm2,\pm4\}\]
                    \[p(\pm 1)=-4,p(2)=2,p(-2)=-10\]
                    \[p(4)=56,p(-4)=-64\]
                    Por lo que es un polinomio irreducible.

                    \item De la misma forma que en el polinomio anterior.
                    \[p(x)=0\implies x\in\{\pm1\}\]
                    \[p(1)=0.5,p(-1)=1.5\]
                    Por lo que es un polinomio irreducible.

                    \item Similarmente a los polinomios anteriores, el conjunto de posibles raíces es finito, pero también hay que considerar el caso donde se puede factorizar en dos polinomios de grado dos. 
                    \[p(x)=0\implies x\in\{\pm1\}\]
                    \[p(\pm1)=1\]
                    Vemos que no tiene raíces, por lo que solo nos falta ver que no se puede factorizar, para eso se asume que es factorizable, luego notamos que son monicos y que por lema de Gauss si $p(x)\in\set{Z}[x]$ y es irreducible en $\set{Z}[x]$ entonces es irreducible en $\set{Q}[x]$.
                    \[p(x)=(x^2+bx+c)(x^2+b'x+c')\]
                    \[p(x)=x^4+(b'+b)x^3+(c'+bb'+c)x^2+(b'c+bc')x+cc'\]
                    \[\implies cc'=1,\,b'+b=b'c+bc'=0,\,c'+bb'+c=-1\]
                    \[\implies b=-b',\,c=c'=1\]
                    \[\implies -b^2+2=-1\iff b^2=3\]
                    Pero $b\notin\set{Z}$, por lo tanto $p(x)$ es irreducible en $\set{Z}[x]$ y en $\set{Q}[x]$
                \end{itemize}
            \end{proof}
        
        \item Sea $p$ un numero primo. Dado un numero racional $x\in\set{Q},x\neq 0$, podemos escribir
        \[x=p^r\cdot\frac{a}{b},\quad a,b,r\in\set{Z},\quad \gcd(p, ab)=1\]
        Se define $v_p(x):=r$ y $v_p(0):=\infty$. Sea
        \[O_p=\{x\in\set{Q}:v_p(x)\geq 0\},\quad m_p=\{x\in\set{Q}:v_p(x)>0\}\]
        \begin{enumerate}[label=(\alph*)]
            \item Demuestre que $(O_p,+,\cdot)$ es un anillo cuyo único ideal maximal es $m_p$

            \item Demuestre que $O_p/m_p\cong\set{Z}/p\set{Z}$
            
            \item Sea $I\subset O_p$ un ideal propio. Muestre que existe un entero positivo $n$ tal que
            \[I=\{x\in\set{Q}:v_p(x)\geq n\}\]

            \item Sea $k>0$ un entero. Considere el ideal $I=m_p^k$. Muestre que
            \[O_p/I\cong\set{Z}/p^k\set{Z}\]
        \end{enumerate}
        \begin{proof}
            \
            \begin{enumerate}[label=(\alph*)]
                \item Hay tres partes para esta demostración, primero que $O_p$ es un anillo, segundo que $m_p$ es ideal y tercero que es el unico ideal maximal.
                \begin{enumerate}
                    \item Sea $x,y\in O_p$, sin perdida de generalidad de asume que $r\geq r'$
                    \[x=p^r\cdot\frac{a}{b},\,y=p^{r'}\cdot\frac{a'}{b'}\]
                    \[\therefore x+ y=p^{r'}\left(p^{r-r'}\frac{a}{b}+ \frac{a'}{b'}\right)=p^{r'}\left(\frac{p^{r-r'}ab'+a'b}{bb'}\right)\]
                    Vemos que $r'\geq 0$, por lo que $x+y\in O_p$.Y claramente se ve lo siguiente
                    \[\gcd\left(p,bb'\left(p^{r-r'}ab'+a'b\right)\right)=1\]
                    Notamos que la suma esta bien definida, y ya que es un subanillo de $\set{Z}$ es conmutativa, sobre esto es fácil ver que para todo elemento del subanillo el inverso de la suma pertenece.
                    \[0=p^r\cdot 0\]
                    \[\implies 0\in O_p\]
                    Por lo que $(O_p,+)$ es grupo abeliano. Luego tomamos los mismos $x,y$.
                    \[x\cdot y=p^{rr'}\frac{aa'}{bb'}\]
                    \[\gcd(p,ab)=\gcd(p,a'b')=1\implies\gcd(p,aa'bb')=1\]

                    \item Se toma $x,y\in m_p$, Similarmente a la parte anterior se nota que $x+y\in m_p$. Luego sea $x\in O_p, y\in m_p$.
                    \[x\cdot y=p^{r+r'}\frac{aa'}{bb'}\]
                    Como $r,r'\geq 0$ y $r>0$, $r+r'>0$. Por lo que es ideal.

                    \item Se asume existe un ideal $M$ tal que $m_p\subset M\subset O_p$. Recordamos la definición de $O_p$ y de $m_p$, con lo que notamos que si $x\in O_p\wedge x\notin m_p\implies v_p(x)=0$ que a su ves implica lo siguiente:
                    \[x=\frac{a}{b}\quad \gcd(p,ab)=1\]
                    Ahora tomamos $x^{-1}$ el cual claramente pertenece a $O_p$. Luego ya que $M$ es ideal $x\cdot x^{-1}=1\in M$ lo que implica que $M=O_p$, lo cual es una contradicción, por ende $m_p$ es ideal maximal. Dado esto asumimos que existe otro ideal maximal $M$ tal que $M\neq m_p$, por lo tanto existe $x\in M\wedge x\notin m_p$, pero ya notamos que los únicos elementos que no pertenecen a $m_p$ son los que cumplen $v_p(x)=0$ y si estos pertenecen a un ideal, el ideal es todo el anillo. Por lo que $m_p$ es un ideal maximal y es único.
                \end{enumerate}

                \item Para demostrar esto usaremos el primer teorema de isomorfismo, y tomaremos el morfismo natural $\func{\varphi}{O_p}{\set{Z}/p\set{Z}}{}{}$ que cumple con lo siguiente:
                \[x=p^r\cdot\frac{a}{b}\quad\gcd(p,ab)=1\]
                \[\varphi(p)=\bar{0}\]
                \[\varphi(a)=\bar{a}\]
                \[\varphi\left(\frac{1}{b}\right)=\bar{b}^{-1}\]
                Este ultimo esta bien definido ya que se sabe que todo elemento no cero en $\set{Z}/p\set{Z}$ tiene inverso y $\gcd(b,p)=1$ por lo que $\bar{b}\neq\bar{0}$, luego vemos que la suma esta bien definida de la siguiente forma:
                \[\varphi\left(\frac{a}{b}+\frac{c}{d}\right)=\varphi\left(\frac{ad+bc}{bd}\right)\]
                \[\varphi\left(\frac{a}{b}+\frac{c}{d}\right)=(\bar{a}\bar{d}+\bar{b}\bar{c})(\bar{b}\bar{d})^{-1}\]
                \[\varphi\left(\frac{a}{b}+\frac{c}{d}\right)=\bar{a}\bar{b}^{-1}+\bar{c}\bar{d}^{-1}\]
                Por el otro lado:
                \[\varphi\left(\frac{a}{b}\right)+\varphi\left(\frac{c}{d}\right)=\bar{a}\bar{b}^{-1}+\bar{c}\bar{d}^{-1}\]
                Por lo que la suma esta bien definida. Ahora facilmente vemos que $\ker\varphi=m_p$. Por esto $O_p/m_p\cong\set{Z}/p\set{Z}$.

                \item Tomamos un $k$ tal que $(k)=I$:
                \[\therefore k=p^r\cdot\frac{a}{b}\]
                Luego por ser un ideal si tomamos un $x\in O_p\implies xk\in I$
                \[xk=p^{r+r'}\cdot\frac{aa'}{bb'}\]
                Sabemos que $r'\geq 0\implies r+r'\geq r$, lo que nos deja que ver que $\forall a\in I: v_p(a)\geq r$, ya que $x$ era un elemento cualquiera. Pero esto nos dice que $I\subseteq\{x\in\set{Q}:v_p(x)\geq r\}$, por lo que nos falta la otra contención. Para ello notamos que 

                \item Usaremos un morfismo similar al de 2. (b) y el primer teorema de isomorfismo.
                \[\varphi(p^k)=0,\varphi(p^r)\neq 0\quad 0\leq r<k\]
                \[\varphi\left(\frac{1}{a}\right)=\bar{b}^{-1}\]
                Este ultimo existe ya que $\gcd(p^k,b)=1$, por lo que $\bar{b}$ tiene inverso en $\set{Z}/p^k\set{Z}$. Dado esto notamos trivialmente que $\ker\varphi=m_p^k$, por lo que podemos concluir que $O_p/m^k_p\cong \set{Z}/p^k\set{Z}$
            \end{enumerate}
        \end{proof}

        \item \textit{Dos maneras de construir} $\set{Q}[i]$. Considere el anillo de los enteros de Gauss:
        \[\set{Z}[i]=\{a+bi:a,b\in\set{Q}\}\]
        \begin{enumerate}[label=(\alph*)]
            \item Demuestre que $\set{Z}[i]$ es un dominio de integridad y que
            \[Frac(\set{Z}[i])=\{a+ib:a,b\in\set{Q}\}\]
            A este ultimo cuerpo le denotamos $\set{Q}[i]$

            \item Demuestre que $\set{Q}[x]/(x^2+1)\cong\set{Q}[i]$. Indicación: encuentre primero un morfismo apropiado $\set{Q}[x]\rightarrow\set{Q}[i]$

            \item Demuestre que $\set{Z}[x]/(x^2+1)\cong\set{Z}[i]$

            \item Demuestre que para todo ideal maximal $I\subseteq\set{Z}[i]$ el cuerpo $\set{Z}[i]/I$
        \end{enumerate}

        \begin{proof}
            \
            \begin{enumerate}[label=(\alph*)]
                \item Hay que demostrar dos cosas, primero que $\set{Z}[i]$ es un dominio de enteros y que $Frac(\set{Z}[i])=\{a+ib:a,b\in\set{Q}\}$
                \begin{enumerate}
                    \item Se asume que existen dos elementos $a+ib$ y $c+id$ distintos de cero, tal que su multiplicación es igual a cero.
                    \[(a+ib)(c+id)=(ac-bd)+i(ad+bc)=0\]
                    \[\implies ac=bd, ad=-bc\]
                    \[\iff acd=bd^2,acd=-bc^2\]
                    \[\iff b(c^2+d^2)=0\]
                    Si $c^2+d^2=0\implies c=d=0\implies c+di=0$, pero dijimos que era un elemento no cero. Por lo que $b=0$
                    \[\implies ac=0,ad=0\]
                    \[\implies (a=0\wedge c=0)\vee(a=0\wedge d=0)\]
                    Si $a=0\implies a=b=0\implies a+ib=0$, pero dijimos que era un elemento no cero. Por lo que $c=d=0$, pero esto es la contradicción mencionada anteriormente, por lo que $\set{Z}[i]$ es dominio.

                    \item Primero recordamos que $Frac(\set{Z}[i])=\{\frac{a}{b}:a,b\in\set{Z}[i]\vee b\neq 0\}$. Primero notamos lo siguiente:
                    \[\frac{a+ib}{c+id}=\frac{(ac-bd)}{c^2+d^2}+i\frac{(ad+bc)}{c^2+d^2}\]
                    Ya que $c+id\neq 0$, $c^2+d^2\neq 0$. Lo que implica que cada termino por separado pertenece a $\set{Q}$, por lo que $Frac(\set{Z}[i])\subseteq\set{Q}[i]$. Luego vemos lo siguiente:
                    \[\frac{a}{b}+i\frac{c}{d}=\frac{ad+ibc}{bd}\]
                    Ya que $b\neq 0\vee d\neq0$, $bd\neq 0$. Y $ad+ibc,bd\in\set{Z}[i]$ por lo que $Frac(\set{Z}[i])=\set{Q}[i]$.
                \end{enumerate}

                \item Para esta demostración se puede usar el primer teorema de isomorfismo, tomando el siguiente morfismo:
                \[\func{\varphi}{\set{Q}[x]}{\set{Q}[i]}{}{}\]
                \[a\mapsto a\]
                \[x\mapsto i\]
                Notamos que $(x^2+1)\subseteq\ker\varphi$, y que $\Ima\varphi=\set{Q}[i]$, por lo que nos queda demostrar la otra contención. Luego, sea $p\in\ker\varphi\setminus\{0\}$.
                \[\implies\varphi(p)=0\]
                \[p=\sum^n_{j=0}a_jx^j\]
                \[\therefore\varphi(p)=\sum^n_{j=0}\varphi(a_j)\varphi(x)^j\]
                \[\varphi(p)=\sum^n_{j=0}a_ji^j=0\]
                Si $n$ es par (en caso de $n$ impar es análogo):
                \[\implies \sum^{n/2}_{j=0}a_{2j}i^{2j}=0\]
                \[\implies \sum^{n/2}_{j=1}a_{2j-1}i^{2j-1}=0\]
                \[\therefore p(x)=q(x)+r(x)\]
                Donde $q$ tiene los coeficientes pares y $r$ tiene los coeficientes impares de $p$.
                \[\implies r(i)=q(i)=0\]
                \[\implies x^2+1\mid r,x^2+1\mid q\]
                \[\implies x^2+1\mid p\]
                \[\implies p\in (x^2+1)\]
                Por lo que $(x^2+1)=\ker\varphi$, lo que implica que $\set{Q}[x]/(x^2+1)\cong\set{Q}[i]$ por el primer teorema de isomorfismo.

                \item De la misma forma que en la demostración anterior $\set{Z}[x]/(x^2+1)\cong\set{Z}[i]$.

                \item Recordamos de Algebra Abstracta I, que $\set{Z}[i]$ es un dominio Euclideano\cite{artin2011algebra}, por lo que ademas es DIP. Dado esto pasan dos cosas, primero $I=(a)$ para algún $a\in\set{Z}[i]$, mas específicamente todo ideal maximal es un ideal primo, y segundo hay una norma definida en $\set{Z}[i]$ la cual llamaremos $N$ este también es aplicable sobre $\set{Z}[i]$.
                \[\therefore \set{Z}[i]/I=\set{Z}[i]/(p)\]
                Donde $p$ es primo de Gauss que genera el ideal $I$. Luego usando el algoritmo de la division notamos la norma de los restos de la division es siempre menor a la norma del divisor y que la norma de cualquier elemento siempre es mayor o igual a 0. Esto nos lleva a notar que hay cantidad finita de elementos en $\set{Z}[i]$ tal que su norma sea menor a la de $p$, ya que el subconjunto de . Por ende $\set{Z}[i]/I$ es finito.
            \end{enumerate}
        \end{proof}
    \end{enumerate}
    
    \bibliographystyle{unsrt}
    \bibliography{bib}
\end{document}